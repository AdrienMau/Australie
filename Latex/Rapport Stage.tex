\documentclass{article}
 
\usepackage[latin1]{inputenc}
\usepackage[T1]{fontenc}
\usepackage[francais]{babel}
\usepackage{lmodern}
\usepackage{stmaryrd}
\usepackage{amsthm}
\usepackage{amsmath}
\usepackage{amssymb}
\usepackage{mathrsfs}
\usepackage{soul}
\usepackage{layout}
\usepackage{setspace}
\usepackage[top=2cm, bottom=2cm, left=3 cm, right=3 cm]{geometry}
\usepackage{pgf,tikz}
\usepackage{graphicx}
\usepackage{xcolor}
\usetikzlibrary{arrows}
\usepackage{listings}
\usepackage{enumitem}
\usepackage{hyperref}
\usepackage{cases}
\usepackage{mathabx}
\usepackage{wrapfig}
%\usepackage{slashbox}
\usepackage{textcomp}
\usepackage{adforn}
\usepackage{multicol}
%\usepackage{fourier} 



\frenchbsetup{StandardLists=true}

\newtheorem*{rem.}{Remarque} %% � utiliser



\newcommand{\ip}{$i_{photodiode}$ }
\newcommand{\fl}{$F_{Lampe}$ }




\begin{document}

\newpage

MAU Adrien - BRIOSNE FREJAVILLE Cl�mence

\bigskip

\bigskip

\begin{center}
\textsc{\LARGE{ Internship report}}\\[1.5cm] 
\end{center}


\bigskip


\section*{Introduction}

Goal: create the smallest holes on SiN membranes.
What is used: FIB + SEM
Chat we have done: influence of ....


\section{Coating}

Different coating have been used: Cr and Au.

Moreover, the thickness of the coating may influence the quality of imaging and milling.
We used different thickness:

\begin{itemize}
\item 10 nm Cr on a 100nm -thin membrane
\item 5 nm Cr on a 30nm -thin membrane
\item 2 nm Au on a 10nm -thin membrane 
\end{itemize}

\textit{Au better for imaging ?}

\section{Milling holes}

In order to mill some holes, the IonLine device for Focused Ion Beam (FIB) has been used. For all experiments, we used a Ga-ion FIB. Moreover, different column parameters such as the current and the aperture can be cleverly chosen.

The influence of the number of loops and the total amount of charges on the milling of holes has been observed.

Theses kind of holes have been obtained:

\bigskip

\textit{Ajouter images / valeur de diametres / total current / loops}


\section{Closing holes}
Smaller holes can be achieved thanks to Scanning Electrons Microscopy (SEM). 
Once again, many parameters can be chosen: 

\begin{itemize}
\item the cycle time (we chose the 4th mode: ...)
\item the size of the scanning area
\item the magnification
\item the number of scans or the total duration of the scanning
\item to make pauses or not
\end{itemize}






\section{Measuring the hole size}






















\end{document}